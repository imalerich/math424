\documentclass[12pt]{jhwhw}
\author{Ian Malerich}
\title{Math 424 - Lab 09}
\usepackage{amssymb, amsfonts, mathtools, graphicx, breqn}
\usepackage{minted, subfig, float, scrextend, setspace}
\usemintedstyle{friendly}

\DeclarePairedDelimiter\ceil{\lceil}{\rceil}

\onehalfspacing
\begin{document}
\raggedright

%% Problem 1
\textbf{1.}
	What happens if in the pseudo-code we discussed in class for tree-structured global
	sum is run when the number of cores is not a power of two? Can you modify the pseudo-code
	so that it will work correctly regardless of the number of cores?
\textcolor[RGB]{240,240,240}{\rule{\textwidth}{0.5pt}}\bigbreak

	\begin{addmargin}[1em]{}
		If we have our number of cores as a power of 2, then each core is assigned
		a unique bit of data and each step reduces our problem space by half.
		Each step requires half as many cores, thus, if we want to have an integer number
		of cores at each step, we must start with our number of cores as a power of 2.
		\bigbreak
		If the number of cores is not a power of 2, we could pretend like we have
		the next power of 2 cores running, if a core which doesn't actually exist 
		is supposed to send data to be added by another core which does exist,
		the actual core will simply assume a 0 was sent. Basically, we are extending
		our global sum to include an additional set of theoretically 0's such that
		we end up with a number of cores which is a power of 2.
		The pseudo code would look something like this.
		\bigbreak
		n = number of actual cores \\
		pow(2, ceil(log2(n))) // number of 'virtual' cores \\

		// if core id $< n-1$, take the partial sum as normal \\
		// else (if id = $n-1$), pretend to receive a 0 until it is our turn to send   \\
		// our data to the next core to do addition \\
	\end{addmargin}

%% Problem 2
\bigbreak
\textbf{2.}
	Derive formulas for the number of receives and additions that core 0 carries out using:
	\begin{enumerate}
		\item the original pseudo-code for a global sum
		\item the tree-structured global sum
	\end{enumerate}
\textcolor[RGB]{240,240,240}{\rule{\textwidth}{0.5pt}}\bigbreak

	\begin{addmargin}[1em]{}
		\begin{enumerate}
			\item A total of $n$ operations are needed, where $n$ is the number of
				items to be added.
			\item The depth of the tree will be $\log_2(n)$ but this includes
				the initial value owned by core 0. Thus it receives and adds an additional
				$\log_2(n)-1$ values to it's original value.
		\end{enumerate}
	\end{addmargin}

%% Problem 3
\bigbreak
\textbf{3.}
	The first part of global sum example - when each core adds its assigned computed values -
	is usually considered to be an example of data-parallelism, while the second part of the
	first global sum - when the cores send their partial sums to the master core, which adds them
	- could be considered to be an example of task-parallelism. What about the second part
	of global sum - when the cores use a tree-structure to add
	their partial sums? Is this an example of data- or task-parallelism? Why?
\textcolor[RGB]{240,240,240}{\rule{\textwidth}{0.5pt}}\bigbreak

	\begin{addmargin}[1em]{}
		This is an example of data-parallelism, each core will be performing the same
		task (i.e. the same block of code), the only distinguishing factor is the tree
		node (the data) to which each core is assigned.
	\end{addmargin}

\end{document}
