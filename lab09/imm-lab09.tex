\documentclass[12pt]{jhwhw}
\author{Ian Malerich}
\title{Math 424 - Lab 09}
\usepackage{amssymb, amsfonts, mathtools, graphicx, breqn}
\usepackage{minted, subfig, float, scrextend, setspace}
\usemintedstyle{friendly}

\DeclarePairedDelimiter\ceil{\lceil}{\rceil}

\onehalfspacing
\begin{document}
\raggedright

%% Problem 1
\textbf{1.}
\textcolor[RGB]{240,240,240}{\rule{\textwidth}{0.5pt}}\bigbreak

	\begin{addmargin}[1em]{}
		Assume we have cores labeled $0,1,2,\ldots (p-1)$, each index denoted by the variable
		my\_rank. We would likely to denote $n$ integers between these $p$ cores.
		Let $k = \ceil{\frac{n}{p}}$, note that $k*p$ need not equal $n$ if $p$ does
		not evenly divide $n$, in this case, all cores will be given equal work of $k$ items
		with the exception of the last core, who will only take the remaining. \\
		Thus, given core with index my\_rank\ldots \\
		case my\_rank$ < (p-1)$: \\
		\begin{addmargin}[2em]{}
			Note that in the loop my\_last\_i is exclusive, while my\_first\_i is inclusive,
			therefore, my\_last\_i should be equal to my\_first\_i for the next core. \\
			my\_first\_i $= i \times k$ \\
			my\_last\_i $= (i+1) \times k$ \\
		\end{addmargin}
		Note that we are adding an extra 1 to my\_rank to convert from the 0 based indecies
		given by the core id to the 1 based indecies I used for my algebra. 
		case my\_rank$ = (p-1)$: \\
		\begin{addmargin}[2em]{}
			This is the last core, we don't want to dip out of bounds of our array, thus
			simply terminate it with value $n$. As the array is 0 based, and my\_last\_i
			is exclusive, this is the value we want and not $n-1$. \\
			my\_first\_i $= i \times k$ \\
			my\_last\_i $= n$ \\
		\end{addmargin}
	\end{addmargin}

%% Problem 2
\bigbreak
\textbf{2.}
\textcolor[RGB]{240,240,240}{\rule{\textwidth}{0.5pt}}\bigbreak

	\begin{addmargin}[1em]{}
		Note that each subsequent core takes twice as long as the previous (given equal
		amounts of work), thus if we desire to have them take equal amounts of time,
		each subsequent core should do half as much work. \\
		Thus we want $\Sigma_{i=0}^{p-1} \frac{k}{2^i} = n$ for some constant $k$, the
		amount of work performed by the initial unit (note that $\frac{k}{2^0} = k$).
		Doing some simple algebra we find $k = n / (\Sigma_{i=0}^{p-1} \frac{1}{2^i})$.
		Note that $k\in \mathbb{R}$, we will take the ceiling as before at each step.
		Then, my\_first\_i = $\ceil{\sum_{i=0}^{\text{my\_rank}}\frac{k}{2^i}}$ and
		my\_last\_i = $\ceil{\sum_{i=0}^{\text{my\_rank+1}}\frac{k}{2^i}}$ with the exception
		for my\_rank$=(p-1)$ then my\_last\_i = n.
		For example, if we had an infinite number of cores, $k=\frac{1}{2}$.
		 
	\end{addmargin}

\end{document}
