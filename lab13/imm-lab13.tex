\documentclass[12pt]{jhwhw}
\author{Ian Malerich}
\title{Math 424 - Lab 13}
\usepackage{amssymb, amsfonts, mathtools, graphicx, breqn}
\usepackage{minted, subfig, float, scrextend, setspace}
\usemintedstyle{friendly}

\DeclarePairedDelimiter\ceil{\lceil}{\rceil}

\onehalfspacing
\begin{document}
\raggedright

%% Chapter 3 starts on page 104
%% Exercises start on page 161

\textbf{3.1}
	What happens in the greeting program if, instead of strlen(greeting)+1,
	we use strlen(greeting) for the length of the message being sent
	by processes $1,2,\ldots,$comm_sz-1? What happens if we use MAX_STRING instead?
	Can you explain these results?
\textcolor[RGB]{240,240,240}{\rule{\textwidth}{0.5pt}}\bigbreak

	\begin{addmargin}[1em]{}
		If we forget to add 1 to strlen(..) we are not sending the null terminating
		character of the string in the message, thus there are no guarantees on
		where the string will terminate when print it out in process 0. \\
		If we use MAX_STRING, we are still sending the null terminator, thus
		printf will work as expected, however this time we are simply sending
		more data than is needed, including a bunch of garbage data after the string. \\
	\end{addmargin}

\textbf{3.2}
\textcolor[RGB]{240,240,240}{\rule{\textwidth}{0.5pt}}\bigbreak

	\begin{addmargin}[1em]{}
	\end{addmargin}

\textbf{3.3}
\textcolor[RGB]{240,240,240}{\rule{\textwidth}{0.5pt}}\bigbreak

	\begin{addmargin}[1em]{}
	\end{addmargin}

\textbf{3.4}
\textcolor[RGB]{240,240,240}{\rule{\textwidth}{0.5pt}}\bigbreak

	\begin{addmargin}[1em]{}
	\end{addmargin}

\end{document}
