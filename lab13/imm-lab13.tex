\documentclass[12pt]{jhwhw}
\author{Ian Malerich}
\title{Math 424 - Lab 13}
\usepackage{amssymb, amsfonts, mathtools, graphicx, breqn}
\usepackage{minted, subfig, float, scrextend, setspace}
\usemintedstyle{friendly}

\DeclarePairedDelimiter\ceil{\lceil}{\rceil}

\onehalfspacing
\begin{document}
\raggedright

%% Chapter 3 starts on page 104
%% Exercises start on page 161

\textbf{3.1}
	What happens in the greeting program if, instead of strlen(greeting)+1,
	we use strlen(greeting) for the length of the message being sent
	by processes $1,2,\ldots,$comm_sz-1? What happens if we use MAX_STRING instead?
	Can you explain these results?
\textcolor[RGB]{240,240,240}{\rule{\textwidth}{0.5pt}}\bigbreak

	\begin{addmargin}[1em]{}
		If we forget to add 1 to strlen(..) we are not sending the null terminating
		character of the string in the message, thus there are no guarantees on
		where the string will terminate when print it out in process 0. \\
		If we use MAX_STRING, we are still sending the null terminator, thus
		printf will work as expected, however this time we are simply sending
		more data than is needed, including a bunch of garbage data after the string. \\
	\end{addmargin}

\textbf{3.2}
	Modify the trapezoidal rule so that it will correctly estimate the integral
	even if comm_sz doesn't evenly devide $n$. (You can still assume $n\geq$comm_sz.)
\textcolor[RGB]{240,240,240}{\rule{\textwidth}{0.5pt}}\bigbreak

	\begin{addmargin}[1em]{}
		\inputminted{c}{3.2.c}
	\end{addmargin}

\textbf{3.3}
	Determine which of the variables in the trapezoidal rule program are local and which 
	are global.
\textcolor[RGB]{240,240,240}{\rule{\textwidth}{0.5pt}}\bigbreak

	\begin{addmargin}[1em]{}
		MPI runs main $n$ times as specified on mpiexec, thus each process has it's own local copy
		of all variables declared in main. Because we are not using any other values, they aren't
		physically shared across each process. However, due to their setup, values such as 
		\textit{a,b,h,local_n} will be equivalent across all processes (and are not changed by 
		any processes) but are not technically global.
	\end{addmargin}

\textbf{3.4}
	Modify the program that just prints a line of output from each process (mpi_output.c) so
	that the output is printed in process rank order: process 0s output first, then process 1s,
	and so on.
\textcolor[RGB]{240,240,240}{\rule{\textwidth}{0.5pt}}\bigbreak

	We could follow the strategy provided by the hello world mpi example. Where
	only my_rank 0 does the printing and all others send that process strings. But
	if we want each process to invoke printf, we can control the timing by
	using MPI_Recv to cause a process to wait until the previous process is done
	printing, once that process is done printing, it will invoke MPI_Send to let
	the next process know it's okay to print.

	\begin{addmargin}[1em]{}
		\inputminted{c}{3.4.c}
	\end{addmargin}

\end{document}
